\chapter{測試與驗證}

\begin{figure}[ht]
    \begin{minted}[
        frame=single,
        framesep=2mm,
        baselinestretch=1.2,
        bgcolor=LightGray,
        fontsize=\footnotesize,
        linenos
    ]{cpp}
#include <iostream>
using namespace std;

int sigma() {...}

int main() {
    int result = sigma(3);
    cout << result << '\n';
}
    \end{minted}

    \captionsetup{justification=centering}
    \caption{主函式細節}
    \label{fig:主函式細節}
\end{figure}

\begin{figure}[ht]
    \begin{minted}[
        frame=single,
        framesep=2mm,
        baselinestretch=1.2,
        bgcolor=LightGray,
        fontsize=\footnotesize,
        linenos
    ]{shell}
$ g++ main.cpp -o main.exe && ./main.exe
6
    \end{minted}

    \captionsetup{justification=centering}
    \caption{shell 編譯指令與輸出結果}
    \label{fig:shell 編譯指令與輸出結果}
\end{figure}

此函式遞迴終止條件為當 $n$ 為 $0$ 或 $1$,若欲求得 $3!$,則呼叫 $sigma(3)$,進入函式後,首先第一層 $n = 3 > 1$ 所以回傳 $n + sigma(n − 1)$,即 $3 + sigma(2)$,接著第二層計算 $sigma(2)$,$n = 2 > 1$,所以回傳 $2 + sigma(1)$,接下來到第三層時,$n = 1 \le 1$,符合終止條件 $(n ≤ 1)$,因此回傳 $n$,即 $1$。

$$sigma(3)=3+sigma(2)=3+2+sigma(1)=3+2+1=6$$